\documentclass[12pt,letterpaper]{article}
\usepackage[]{todonotes}
\usepackage[top=1in, bottom=1in, left=1in, right=1in]{geometry}
\presetkeys{todonotes}{inline}{}
\usepackage{times}
\usepackage[backend=bibtex]{biblatex}

\addbibresource{ENGR3510.bib}
\begin{document}
\title{Annotated Bibliography}
\author{ENGR 3510 Project}
\date{\today}
\maketitle

The purpose of this project is to design a force transducer for our wind tunnel that is capable of measuring both lift and drag. Other features may be added as desired by the group.

You can add comments with \todo{CB: Do this}. Make sure to put your initials. It will do weird things to the text; that's fine. 
Citations are made with \cite{Dolan2016} where Dolan2016 could be replaced with any citation key. 

\todo{CB: An annotated bibliography is a list of references with descriptions summarizing their content and importance to the work at hand.}


\begin{enumerate}
	\item \cite{Gerontakos2007} This article studies the affects that vortexes cause on the lift and drag components of different types of wings. This article provides useful information as to what to expect within some force calculations as well as how to prepare the calculations in the first place. However, this document used some extremely technical language that was hard to follow. Furthermore, many of the discussed results and calculations are methods that are not familiar and therefore harder to follow.
	\item \cite{Portman2009} This article mostly explained a type of measuring system that helps measure different forces on an object within a wind tunnel. This is done through exploring the applications of said system as well as issues and fixes with the measurements of forces within a wind tunnel. This article provides some useful information into how the forces within a wind tunnel are measured with current devices and what purpose they are measured for. I found section 1 of the article to be most useful with my current knowledge as it was more of an summary versus an in-depth exploration. There was a decent amount of information that I did not understand, however. I did not understand the discussion on anisotropy within the article. It appeared in multiple sections, and I could not understand what it was or what its impact was. There were also some discussions of bridges and signals that I could not follow in relation to the gauges used in some of the systems.
	\item \cite{Nan2013} The author writes about the components of wind tunnels and their purpose in drag force measurements. The paper discusses different ways drag force is measured, such as floating-element on oil and load cell measuring. The different styles of drag force measuring can be useful to know from this paper. However, load cell measuring is a little tricky to understand and I have not yet grasped the concept. 
	
	
	\item \cite{Gonzalez} This article talks about the three main components of wind tunnel balances, External, internal, and Rotary. A useful part of this document is the it talks some about calibration in wind tunnels. There is one part about data gathering equipment in wind tunnels, it would be useful if these were expounded upon in detail.  
	\item \cite{Wilson-strain-gage} This reference explains how strain gages work. Since our solution will likely need a strain gage, we'll start here for looking at material.

	\item \cite{Hubova2019} This article deals with the evaluation parts of wind loads on various layouts of plastic and steel mesh fabrics.I do not know what the Reynolds number is, I guess it is talking about his data from the impermeable surface by reduction.
I do understand how to calculate the basic wind velocity.
	\item \cite{Yan2009} It is mainly an experiment, whereas a number of aerodynamic research is done, mainly by measuring the forces acted upon the pressure distribution, the visualization of the flow and more like the test models. 
In this article, the force balance is what I do not understand because, how are there so many different forces acting upon an airplane?
I do understand Force measurement is the basic and conventional function that for a wind tunnel, it is important that new high-speed vehicles designed in air or ground or water, so that we as designers can see how much pressure or can be used against it, or how strong it is.
	\item \cite{robert2012} This article mainly is about an electronic engineer at Glenn Research Center who requested NASA engineering and safety center to provide technical support for an evaluation of an existing force measurement system.
This also talks about the steps for calibrating the force balance system.
I don’t understand how the actually calculated anything because they just used excel
I do understand how they calculated the Lift coefficient (CL) because it gives an initial formula and in order to find the lift coefficient, they used the partial derivative to find the absolute standard systematic uncertainty in  CL also the same process was done to find the Drag coefficient and the Force coefficient.




\end{enumerate}

\printbibliography

\end{document}

