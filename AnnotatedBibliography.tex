\documentclass[12pt,letterpaper]{article}
\usepackage[]{todonotes}
\usepackage[top=1in, bottom=1in, left=1in, right=1in]{geometry}
\presetkeys{todonotes}{inline}{}
\usepackage{times}
\usepackage[backend=bibtex]{biblatex}
\addbibresource{ENGR3510.bib}
\begin{document}
\title{Annotated Bibliography}
\author{ENGR 3510 Project}
\date{\today}
\maketitle

You can add comments with \todo{CB: Do this}. Make sure to put your initials. It will do weird things to the text; that's fine. 
Citations are made with \cite{Dolan2016} where Dolan2016 could be replaced with any citation key. 

\todo{CB: An annotated bibliography is a list of references with descriptions summarizing their content and importance to the work at hand.}

\begin{enumerate}
	\item \cite{Nan2013} The author writes about the components of wind tunnels and their purpose in drag force measurements. The paper discusses different ways drag force is measured, such as floating-element on oil and load cell measuring. The different styles of drag force measuring can be useful to know from this paper. However, load cell measuring is a little tricky to understand and I have not yet grasped the concept. 
	
	
	\item \cite{Gonzalez} This article talks about the three main components of wind tunnel balances, External, internal, and Rotary. A useful part of this document is the it talks some about calibration in wind tunnels. There is one part about data gathering equipment in wind tunnels, it would be useful if these were expounded upon in detail.  
\end{enumerate}

\printbibliography

\end{document}